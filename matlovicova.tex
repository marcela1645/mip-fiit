\documentclass{acm_proc_article-sp} % alebo sig-alternate
\usepackage[english]{babel}
\usepackage{url}
\usepackage{booktabs}
\setlength{\heavyrulewidth}{1.2pt}
\setlength{\lightrulewidth}{0.7pt}

\begin{document}



\title{Dynamically Selecting an Appropriate Context Type
for Personalisation}


\numberofauthors{2}

\author{
\alignauthor
Tomas Kramar\\
       \affaddr{Faculty of Informatics and Information}\\
       \affaddr{Technologies}\\
       \affaddr{Slovak University of Technology}\\
       \affaddr{Bratislava, Slovakia}\\
       \email{kramar@fiit.stuba.sk}
\alignauthor
Mária Bieliková\\
       \affaddr{Faculty of Informatics and Information}\\
       \affaddr{Technologies}\\
       \affaddr{Slovak University of Technology}\\
       \affaddr{Bratislava, Slovakia}\\
       \email{bielik@fiit.stuba.sk}
}

\maketitle

\begin{abstract}
Narrowing down the context in the ranking phase of infor-
mation retrieval has been shown to produce results that are
more relevant to searcher's need. We have identiced two
types of contexts that could be used in the process of per-
sonalisation. We research these contexts in the domain of
personalised search, but show that our approach can be used
for any kind of personalisation or recommendation. We fo-
cus on two aspects of the context: temporal context and
activity-based context and describe a more general person-
alisation framework based on lightweight semantics, that can
leverage any type of context.
\end{abstract}


\category{H.3.3}{information search and retrieval}{Information filtering }


\section{Introduction}

Recommendation systems and search engines play a cru-
cial role in accessing the amount of content that can be
found of the Web nowadays. Users are usually able to fil-ter the information by issuing short keyword queries, but
this model has several known disadvantages. The number
of keywords is usually low, typically 1-3 keywords [8] and
many of the words are ambiguous. The queries are almost
never accurate [5], they are either too generic or too specific,but almost never exactly aligned with the specific intent theuser has in mind. When we combine the impact of each of
the described problem, we come to a conclusion, that finding
the relevant document is indeed a diffcult task, both for the
user and the search engine. \\ 
A number of approaches have been researched, each with
the specific goal of helping people and relevant content,
preferably without changing the established paradigm of
searching by keywords. The approaches range from modify-
ing the query in-place to better capture the specialc intent
\\
\\
\small
Permission to make digital or hard copies of all or part of this work for
personal or classroom use is granted without fee provided that copies are
not made or distributed for profit or commercial advantage and that copies
bear this notice and the full citation on the first page. To copy otherwise, to
republish, to post on servers or to redistribute to lists, requires prior specific
permission and/or a fee.\\
RecSys'12, September 9-13, 2012, Dublin, Ireland, UK. \\
Copyright 2012 ACM 978-1-4503-1270-7/12/09 ...\$15.00.
\normalsize
\\
\\
the user has in mind, i.e., query reformulation [3], learning
to rank [9], or relevance feedback [12].
Each of the existing approaches leverages some sort of con-
text to infer user's search intent and personalize the search
results accordingly. Most of the time, the search context
is long-term, built from all available data, e.g. learning to
rank approaches use clickthrough data to learn user's pref-
erence based on her past search result clicks. The other
extreme, short-term context is used in relevance feedback,
where search results are modified according to user's last ac-
tion, with a basic assumption, that a click represents a form
of implicit feedback on the relevance of the clicked search
result.
\\
 In all cases, the time by which the context isbound is
determined statically and does not adapt to what is ap-
propriate in the given situation. Long-term search context
has the advantage that much data about the user is avail-
able, so the personalisation system can make reliable model
of user's long-term interests. However, when a new, previ-
ously unseen goal appears, the information in the long-term
model can shadow it, leading to bad personalization. E.g.,
an aquarium guide looking to buy a new Barracuda hard
drive - the term Barracuda is associated with a sea and fish the model of his interests and the search results are person-
alized towards results that deal with sea and fish, while the
guide is in fact looking for information on the specific model
of the hard drive. On the other hand, the short-context can
contain so little information, that no personalisation is pos-
sible, because no search goal can be inferred. We believe
that the solution to this problem is to select the size of con-
text dynamically at the time of search, and that the search
context should be as specific as possible.
\\
 We have identified two types of search context that may be helpful in personalizing the search:
 \begin{itemize}
\item
 Activity-based context, which is based   on user's actions;
 what is she doing, how is she interacting with the search engine in terms of clicking on the search results.The activity-based search context is basically bounded by the changes in information-seeking goals. This con-text starts, and ends with each change in information-seeking goal.
\item 
 Recurrent temporal context, which is based on user's
 preferences in time. We assume that users form some
 kinds of recurrent habits that correlate with time, e.g.,
 someone may be searching for work-related goals most
 of the time during work-week, while on weekends, the
 search goals are very different. The temporal habits may happen on a larger scale, e.g. searching for aspara-gus recipes in spring, during the asparagus season.
 \end{itemize}
\section{CONTEXT MODELING AND PERSONALISATION
	FRAMEWORK}
We model user's context using lightweight metadata ex-
tracted from accessed documents. Normally, a personalisa-
tion system would have access to this type of information
in a direct way, but for the purposes of our research, we are
collecting this information using a logging proxy server [1].
Usage of the proxy server is completely transparent for the
users and does not affect system performance. Users have to
explicitly opt-in for the proxy and they are aware that their
activity is logged and we provide tools to selectively remove
logged data or opt-out.
\\
For each accessed document we extract document meta-
data, along with other user interaction indicators (mouse
and keyboard activity and time spent on page), which are
used to calculate implicit feedback on the page.
The context model is a hypergraph H :=< V;E >, with
a set of vertices V = A [ P [ T, where A represents a
set of users accessing the pages: A = (a1; a2;  ; ak), P
represents a set of pages P = (p1; p2; pl) and T repre-
sents a set of terms T = (t1; t2; tm); and a set of edges
E = (a; p; t)ja 2 A; p 2 P; t 2 T and P \ T = ;;A \ P =
;;A \ T = ;. Using this representation is advantageous, as
it allows for good denormalization and allows us to track
each of the vertices independently. It may seem intuitive
to merge accesses and pages, but this models allows us to
abstract page from access, and if the document represented
by the URL (page) changes, we can create new vertex in the
graph and connect it respectively.
\\
We use following series of stepsto produce an interest
model that can be used for recommendation or personalisation.
\begin{itemize}
 \item
 First, search context can be selected by any arbitrary
 method. Context selection is based on selection of ac-
 tions, e.g., we can use a long-term context and select all
 documents ever viewed by the user, or we can leverage
 activity based search context and select only the doc-
 uments that she viewed for the purposes of her current
 information goal.
 \item
 Next, we extract a subgraph Hc from the original hy-
 pergraph H, by selecting only vertices Ac 2 A, that
 match the context restriction established in previous
 step. The subgraph also contains vertices adjacent to
 Ac, i.e., Pc 2 P and Tc 2 C and their connecting edges
 Ec 2 E.
 \item
 The subgraph Hc now represents the interest model
 for current context. We further enrich it by nding its
 nearest neighbors in terms of similarity in access pat-
 terns and similarity of the acquired document meta-
 data T. This interest model and models of similar
 interests can now be used for personalisation or rec-
 ommendation.
	
\end{itemize} 	
 To validate this context acquisition approach we ran syn-
thetic experiments [10] and experiments in a live system [11].
In both cases we used a simple form of long-term context,
by using Hc = H. In both cases we evaluated our approachin the domain of personalised search and used two simple
 query reformulation approaches:
  \begin{itemize}
  	\item search the query history in the interest models (ex-
  	tracted from URLs of accesses to search engine, nd
  	reformulation patterns of the current query and se-
  	lect the one with highest satisfaction score, calculated
  	by considering implicit feedback on clicked results, as-
  	suming, that unsatisfactory search results lead to clicks
  	with low implicit feedback scores and to a further query
  	reformulation until the user is satised.
  \item 
  search the metadata in the interest models and nd
  terms that are often co-occurring with the terms from
  the current query. Select the most frequent co-occur-
  rences and extend the original query.  	
 \end{itemize}
 Using this experimental setup, we were able to substan-
 tially increase the relevance of search results and thus val-
 idate our approach. The proposed framework has two im-
 portant properties: both context, and the personalisation
 method can be selected independently and we believe that
 using a different form of context than the long-term con-
 text, one could improve the relevance of recommendations
 even more.
 \section{ACTIVITY-BASED CONTEXT}
 Basic idea of an activity-based context is that informa-
 tion from previously accessed documents can help to under-
 stand the current information goal. E.g., imagine a series of
 queries: watermelon; used cars; bentley; jaguar { it is easy
 	to see that the current query jaguar is connected with two
 	previous queries, used cars and bentley and that the search
 	goal in this case is to buy a car. With this knowledge, the
 	personalisation can be moved towards search results that
 	deal with used jaguar cars.
 	\\
 The problem of finding activity-based search context is
 stated as follows. The user is issuing queries and clicking
 on search results. The goal of activity-based search context
 detection is, for a current query, select the previous queries
 (and their respective clicked search results), that are part
 of the same search goal as the current query. This problem
 is equivalent to the problem of detecting the search session,
 when we define the search session as a set of queries following
 the same goal. Note that the queries that are part of the
 same search session may not be issued sequentially, but can
 be interleaved.
 \\
 The problem of detecting search sessions is well researched,
 although the definition of a search session usually varies.
 The approaches could be categorized into three classes. The
 methods that are based on time, e.g. [6], are not appropri-
 ate for this denition of search session. They assume that a
 search session is bounded by a higher inactivity time, i.e. if
 the query is issued after a specified time from the last query
 elapsed, the query is considered to start a new session. The
 pause in the interaction with search engine can be caused
 by external factors (e.g., phone call, lunch) and user can re-
 turn to the original task afterwards. Similarly, the search
 goal can change rapidly, without a pause in the interaction.
 The other class of approaches is based on lexical similarity
 between queries [7]. Two queries are considered part of the
 same session if they share common terms. However, this
 class of approaches does not deal with semantic similarity,
 e.g., queries information retrieval and IR. The third class of
 \section{RECURRENT TEMPORAL CONTEXT}
 We postulate that there are patterns in user behavior
 changes that are related to time. For example searching
 for summer resorts before summer holidays, or searching for
 which movies to watch on Sunday night. We believe that
 these patterns are recurrent and appear more or less regu-
 larly during the same periods of day, week, month or year.
 To the best of our knowledge, there are no studies that
 would support or disprove our hypotheses. There are many
 works that study the temporal dynamics of search, but all of
 them analyze it only globally, and do not focus on individual
 patterns. For example, in [13] authors analyze transactional
 logs from the Excite search engine using Markov matrices
 and Poisson sampling and explore variations in aggregated
 users' searches related to changes of time in day. Or sim-
 ilarly, Beitzel et al. [2] analyzed a partially manually topi-
 cally labeled large-scale transactional log of a search engine
 and analyzed topical patterns in the queries and results, but
 again, only in an aggregated form.
 \section{CONCLUSIONS AND FUTURE WORK}
 In our work, we try to design a framework for personalisa-
 tion based on context constraints, with the aim of selecting
 the proper type of context dynamically. So far, we have
 designed and evaluated a framework based on lightweight
 semantics and nearest neighbors. We have identified two
 types of context that might be helpful in constraining the
 \section{ACKNOWLEDGMENTS}	

\bibliographystyle{abbrv}
\bibliography{zdroje}

\balancecolumns 

\end{document}
